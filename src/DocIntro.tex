Bijna elk linux-systeem is standaard voorzien van een uitgebreide set van documentatie. De
meeste documentatie bevat de man-pages. `man' is een afkorting voor Manual ofwel handleiding. Dat zal je vaker tegen
gaan komen dat commando's op een Linux en andere Unix-systemen afkortingen zijn. Het afkorten scheelt typen, wat
cruciaal was op de oude stugge toetsenborden van de PDP-systemen waarop Unix is ontworpen. Op deze toetsenborden kreeg
je zere vingers van het typen, dus alles dat typen bespaarde was meegenomen.
